\section{Ejercicio 1}

	\subsection{binf2dec}
		Recibe un arreglo numpy de 16 cifras binarias y lo transforma a punto flotante IEEE 754. Su primera acción es verificar si el arreglo es válido, mientras llama a conv\_bin\_exp y conv\_bin\_mant. En caso de encontrar error escribe en pantalla y no devuelve nada. En caso de ser válido se analiza para casos especiales. Si es NaN, 0 o $\pm\infty$ se devuelve ese valor. En caso de ser un número normal o subnormal se procede a su respectivo cálculo.
		
		\begin{sourcecode}[\label{codigo-python-1}]{python}{Definición de binf2dec}
def binf2dec(arr):
    # Verificación
    if len(arr) != 16:
        print("Por favor ingrese un número binario de 16 dígitos")
        return
    signo = arr[0]											# Calculo signo
    if signo != 0 and signo != 1:							# Reviso si hubo error
        print("Por favor ingrese un número en binario")
        return
    exp = conv_bin_exp(np.array(arr[1:6]))                  # Calculo exponente
    if exp == -1:                                           # Reviso si hubo error
        print("Por favor ingrese un número en binario")
        return
    mantissa = conv_bin_mant(np.array(arr[5 + 1:16]))       # Calculo mantissa
    if mantissa == -1:                                      # Reviso si hubo error
        print("Por favor ingrese un número en binario")
        return

    # Analizo casos
    if exp == 0:
        if mantissa == 0:                                   # Caso 0
            n = 0
        else:                                               # Caso sub-normal
            n = sub_normal(signo, mantissa)
    elif exp == 31:
        if mantissa == 0:                                   # Caso infinito
            n = ((-1)**signo)*np.inf
        else:                                               # Caso not a number
            n = np.nan
    else:
        n = normal(signo, exp, mantissa)                    # Caso normal

    return n
\end{sourcecode}




	\subsection{conv\_bin\_exp}
		Recibe un arreglo que tiene solo la parte exponencial de un número binario. Su función es convertir la parte exponencial binaria en número decimal. Para esto, se hace un for que pase por todas las posiciones del arreglo, por cada 1 del arreglo se suma 2 al exponente de su posición antes de la coma.
		
		\begin{sourcecode}[\label{codigo-python-2}]{python}{Definición de conv\_bin\_exp}	
def conv_bin_exp(arr):
    num = 0
    for i in range(len(arr)):
        if arr[i] == 1:
            num = num + 2 ** ((len(arr) - 1) - i)
        elif arr[i] != 0:
            num = -1
            break
    return num\end{sourcecode}

	\subsection{conv\_bin\_mant}
		Calcula el valor de la mantissa, hace un for y cada vez que encuentra un 1 multiplica por su respectiva potencia de 2, sumando cada resultado hasta llegar a la última cifra y así devolver el valor de la mantissa. En caso de encontrar un número distinto de 0 o 1 se devuelve -1 para significar error.
		
				\begin{sourcecode}[\label{codigo-python-3}]{python}{Definición de conv\_bin\_mant}	
def conv_bin_mant(arr):
    num = 0
    for i in range(len(arr)):
        if arr[i] == 1:
            num = num + 2 ** (-1 - i)
        elif arr[i] != 0:
            num = -1
            break
    return num\end{sourcecode}

	\subsection{normal}
		Recibe el signo, exponente y mantissa para calcular el valor en punto flotante del número normal utilizando la fórmula vista en clase. Devuelve este valor.

\begin{sourcecode}[\label{codigo-python-4}]{python}{Definición de normal}	
def normal(s, exp, mantissa):
    exp = exp - SESGO
    n = ((-1) ** s) * (1+mantissa) * (2 ** exp)
    return n\end{sourcecode}

	\subsection{sub\_normal}
		Recibe el signo y mantissa para calcular el valor en punto flotante del número subnormal utilizando la fórmula vista en clase. Devuelve este valor.

\begin{sourcecode}[\label{codigo-python-5}]{python}{Definición de sub\_normal}	
def sub_normal(s, mantissa):
    exp = 1 - SESGO
    n = ((-1) ** s) * mantissa * (2 ** exp)
    return n\end{sourcecode}

%%%%%%%%%%%%%%%%%%%%%%%%%%%%%%%%%%%%%%%%%%%%%%%%%%%%%%%%%%%%%%%%%%%%%%%%%%%%%%%%%%%%%
\section{Ejercicio 2}

	\subsection{test}
	Función de prueba. Prueba distintos valores posibles a partir de un arreglo que contiene el número en formato punto flotante de 16 bits y su forma decimal. Compara ambas columnas e indica si hubo error. Se consideró incluir en el testeo:
	
\begin{itemize}[label={--}]
				\item Cero
				\item +$\infty$
				\item -$\infty$
				\item Arreglo no binario
				\item Arreglo de menos de 16 bits
				\item Arreglo con texto
				\item Número normal entero
				\item Número normal racional y mayor a 1 
				\item Número normal racional y menor a 1
				\item Número sub-normal
				\item Not a number
			\end{itemize}	
	
En los últimos dos casos tuvimos que tomar criterios de verificación distintos. Con los números sub-normales consideramos que, al ser tan pequeños, habría que tener en cuenta el error de la máquina por lo que verificamos que pertenezca a un intervalo en vez de chequear la igualdad. Para el caso not a number notamos que el testeo general no iba a funcionar nunca entonces usamos la función de numpy np.isnan.

		
		\begin{sourcecode}[\label{codigo-python-5}]{python}{Definición de test}
def test():

	test_arr = [["Cero: |0|0|0|0|0|0|0|0|0|0|0|0|0|0|0|0|",
	             np.array([0, 0, 0, 0, 0, 0, 0, 0, 0, 0, 0, 0, 0, 0, 0, 0]),	   0],
                ["Más infinito: |0|1|1|1|1|1|0|0|0|0|0|0|0|0|0|0|",
                 np.array([0, 1, 1, 1, 1, 1, 0, 0, 0, 0, 0, 0, 0, 0, 0, 0]),  np.inf],
                ["Menos infinito: |1|1|1|1|1|1|0|0|0|0|0|0|0|0|0|0|",
                 np.array([1, 1, 1, 1, 1, 1, 0, 0, 0, 0, 0, 0, 0, 0, 0, 0]), -np.inf],
                ["Non Binary: |0|0|0|0|0|0|0|0|0|0|0|0|0|2|0|0|",
                 np.array([0, 0, 0, 0, 0, 0, 0, 0, 0, 0, 0, 0, 0, 2, 0, 0]), 	None],
                ["Bad size: |0|0|",
                 np.array([0, 0]),                                           	None],
                ["Not even a number: |h|o|l|a|c|o|m|o|e|s|t|a|s|v|o|s|",
                 np.array(["holacomoestasvos"]),                             	None],
                ["Normal (40): |0|1|0|1|0|0|0|1|0|0|0|0|0|0|0|0|",
                 np.array([0, 1, 0, 1, 0, 0, 0, 1, 0, 0, 0, 0, 0, 0, 0, 0]), 	  40],
                ["Normal (-12.84): |1|1|0|0|1|0|1|0|0|1|0|0|0|0|0|0|",
                 np.array([1, 1, 0, 0, 1, 0, 1, 0, 0, 1, 0, 0, 0, 0, 0, 0]),   -12.5],
                ["Normal (0.09375): |0|0|1|0|1|1|1|0|0|0|0|0|0|0|0|0|",
                 np.array([0, 0, 1, 0, 1, 1, 1, 0, 0, 0, 0, 0, 0, 0, 0, 0]), 0.09375],
                ["Sub-Normal (4e-5): |0|0|0|0|0|0|1|0|0|1|0|0|1|0|0|0|",
                 np.array([0, 0, 0, 0, 0, 0, 1, 0, 1, 0, 0, 1, 1, 1, 1, 1]),   4e-05],
                ["NaN: |1|1|1|1|1|1|1|1|1|1|1|1|1|1|1|1|",
                 np.array([1, 1, 1, 1, 1, 1, 1, 1, 1, 1, 1, 1, 1, 1, 1, 1]),  np.nan]]


    # Casos generales
    for i in range(len(test_arr) - 2):
        print("\n" + test_arr[i][0])
        n = binf2dec(test_arr[i][1])
        if(n != test_arr[i][2]):
            print("ERROR")

    # Caso Sub-Normal
    print("\n" + test_arr[len(test_arr) - 2][0])
    n = binf2dec(test_arr[len(test_arr) - 2][1])
    if not (n >= (test_arr[len(test_arr) - 2][2] - 2**-10))
    and ((n <= test_arr[len(test_arr) - 2][2]) + 2**-10):
        print("ERROR")

    # Caso Not a Number
    print("\n" + test_arr[len(test_arr) - 1][0])
    n = binf2dec(test_arr[len(test_arr) - 1][1])
    if not np.isnan(n):
        print("ERROR")

    print("\n" + "Fin del programa de prueba")
    
    return\end{sourcecode}



